% Options for packages loaded elsewhere
\PassOptionsToPackage{unicode}{hyperref}
\PassOptionsToPackage{hyphens}{url}
%
\documentclass[
]{article}
\usepackage{amsmath,amssymb}
\usepackage{iftex}
\ifPDFTeX
  \usepackage[T1]{fontenc}
  \usepackage[utf8]{inputenc}
  \usepackage{textcomp} % provide euro and other symbols
\else % if luatex or xetex
  \usepackage{unicode-math} % this also loads fontspec
  \defaultfontfeatures{Scale=MatchLowercase}
  \defaultfontfeatures[\rmfamily]{Ligatures=TeX,Scale=1}
\fi
\usepackage{lmodern}
\ifPDFTeX\else
  % xetex/luatex font selection
\fi
% Use upquote if available, for straight quotes in verbatim environments
\IfFileExists{upquote.sty}{\usepackage{upquote}}{}
\IfFileExists{microtype.sty}{% use microtype if available
  \usepackage[]{microtype}
  \UseMicrotypeSet[protrusion]{basicmath} % disable protrusion for tt fonts
}{}
\makeatletter
\@ifundefined{KOMAClassName}{% if non-KOMA class
  \IfFileExists{parskip.sty}{%
    \usepackage{parskip}
  }{% else
    \setlength{\parindent}{0pt}
    \setlength{\parskip}{6pt plus 2pt minus 1pt}}
}{% if KOMA class
  \KOMAoptions{parskip=half}}
\makeatother
\usepackage{xcolor}
\setlength{\emergencystretch}{3em} % prevent overfull lines
\providecommand{\tightlist}{%
  \setlength{\itemsep}{0pt}\setlength{\parskip}{0pt}}
\setcounter{secnumdepth}{-\maxdimen} % remove section numbering
\ifLuaTeX
  \usepackage{selnolig}  % disable illegal ligatures
\fi
\IfFileExists{bookmark.sty}{\usepackage{bookmark}}{\usepackage{hyperref}}
\IfFileExists{xurl.sty}{\usepackage{xurl}}{} % add URL line breaks if available
\urlstyle{same}
\hypersetup{
  pdftitle={L\textquotesingle Umanesimo, Machiavelli e Guicciardini},
  hidelinks,
  pdfcreator={LaTeX via pandoc}}

\title{L\textquotesingle Umanesimo, Machiavelli e Guicciardini}
\author{}
\date{}

\begin{document}
\maketitle

\hypertarget{umanesimo}{%
\section{Umanesimo}\label{umanesimo}}

\begin{itemize}
\tightlist
\item
  Nasce tra la fine del 1300 e inizio 1400
\item
  Corrente culturale ottimista, investe un po\textquotesingle{} tutte le
  arti

  \begin{itemize}
  \tightlist
  \item
    2 correnti, filosofia ottimista come l\textquotesingle umanesimo che
    dice "siamo cresciuti e ci siamo civilizzati, andiamo a migliorare e
    progredire traverso la ragione"
  \end{itemize}
\item
  L\textquotesingle uomo viene visto in un modo diverso
\item
  Razionalismo
\item
  Rileggono classici e rivedono un nuovo significato, come faceva
  Petrarca

  \begin{itemize}
  \tightlist
  \item
    donazione di Constantino, secondo la donazione Constantino ha donato
    un pezzo di terra
  \item
    è stato ristudiato ed è falso, Constantino non poteva usare quei
    termini
  \end{itemize}
\end{itemize}

\hypertarget{machiavelli}{%
\section{Machiavelli}\label{machiavelli}}

\begin{itemize}
\tightlist
\item
  Nasce a firenze nel 1469
\item
  Entra a servizio della repubblica di Firenze

  \begin{itemize}
  \tightlist
  \item
    Torturato e esiliato per 1 anno
  \end{itemize}
\item
  1520 viene commissionato dai medici per la storia fiorentina
\item
  Parte dal realismo e classici, perchè secondo lui
  l\textquotesingle esempio dei grandi classici può essere utile per
  risolvere i problemi attuali
\item
  "Il fine giustifica i mezzi"
\item
  Monarchia Italiana, nonostante lui fosse repubblicano
\item
  crede nell\textquotesingle rng anche lui, come Giovanni Boccaccio
\item
  "Il principe", 26 capitoli scritta nel 1513, cerca di indicare la via
  per creare un\textquotesingle italia unita

  \begin{itemize}
  \tightlist
  \item
    Il Papa è il problema
  \item
    spiega che il principe deve comportarsi anche da leone per il bene
    comune
  \end{itemize}
\end{itemize}

\hypertarget{guicciardini}{%
\section{Guicciardini}\label{guicciardini}}

\begin{itemize}
\tightlist
\item
  Nasce a Firenze nel 1483 da una famiglia ricca vicina ai medici

  \begin{itemize}
  \tightlist
  \item
    non viene torturato come Machiavelli, ma viene esiliato
  \end{itemize}
\item
  Nel 1516 lo riassume il Papa
\item
  diventa governatore della città di Modena e Reggio Emilia
\item
  Ebbe la fama di essere incorruttibile
\item
  Sacco di roma

  \begin{itemize}
  \tightlist
  \item
    viene giudicato come uno dei colpevoli e viene allontanato
  \end{itemize}
\item
  Muore nel 1540
\item
  Ogni evento storico è uno e uno solo, quindi deve essere risolto in un
  modo unico, quindi non concorda con Machiavelli
\item
  Visione pessimista
\item
  Monarchico
\item
  Fonti contemporanee
\item
  Primo grande storico della letteratura italiana, in quanto analizza
  con rigore scientifico le fonti
\item
  Primo esempio di prosa scentifica in italia
\end{itemize}

\end{document}
